\documentclass[a4paper]{article}
\usepackage[utf8]{inputenc}
\usepackage[english]{babel}

\title{System Requirements}
\author{Group D}
\date{\today}

\begin{document}

  \tableofcontents

  \section{Methods/techniques}
  %Description of the techniques
  %Why we have chosen these methods, what did we want from the different 
  \subsection{Brainstorming}
  Short on brainstorming - general
    \subsubsection{Description}
    How we used it
    \subsubsection{Why chosen}
    Why we used it
    \subsubsection{Outcome}
    
    \subsection{Stakeholder analysis}
    Brainstormin
    \subsubsection{Description}
    How we used it
    \subsubsection{Why chosen}
    Why we used it
    \subsubsection{Outcome}
    Testar
    vi kom fram till vem vi egentligen gjorde systemet för och det gjorde att vi pratade med kunden och då kom vi fram till att staten skulle vara med och det hade vi inte diskuterat tidigare. 
Contextdiagrammet hjälpte oss att få en överblick på systemet och de olika stakeholders vi har.
    
    \subsection{Similar companies}
    In order to get inspiration and see what there is today
	
    Regional in Stockholm - SL
    Regional in Gothenburg - Västtrafik
    Reginoal in Skåne - Skånetrafiken
    Country trains - SJ
    
    \subsubsection{Description}
    How we used it
    \subsubsection{Why chosen}
    Why we used it
    \subsubsection{Outcome}
    
  \section{Future work}
  %For next release
  Virtual Windows, interview users, questionnaires, suppliers
  
  \section{Reflection}
  %What was succesful/what was challening
  %Example, what have we learned?
  %What could have been done differently?
  %What we have learned related to learning objectives

  Jag ska översätta detta är tanken, Alex.

  Det finns en anledning till varför eliciteringsbarriärer finns, det är inte lätt att ha kunder och att komma fram till en gemensam konsensus.
  Vi har än så länge mest ägnat oss åt brainstorming och diskussioner mellan oss själva och med kunderna. Det är lätt att glömma de olika typerna av elicitering och att använda sig av dem. 
  
  Inte lönt att ta upp för mycket tekniska detaljer med kunderna.

	Än så länge har vi inte haft så stor möjlighet att dela upp arbetet och det är inte jätteeffektivt att sitta i en grupp på 8 personer och arbeta med samma sak, det är lätt att vissa inte blir lika delaktiga än andra. 


  \section{Personal statement}
  %Each member briefly explains its individual contributions to the project results
  
  \subsection{Alexander}
  \subsection{Emma}
  \subsection{Fredrik}
  \subsection{Hannes}
  \subsection{Johan}
  \subsection{Martin}
  \subsection{Mattias}
  \subsection{Philip}

\end{document}
