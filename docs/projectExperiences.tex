\documentclass[a4paper]{article}
\usepackage[utf8]{inputenc}
\usepackage[english]{babel}

\title{System Requirements}
\author{Group D}
\date{\today}

\begin{document}

  \tableofcontents

  \section{Methods/techniques}
  %Description of the techniques
  %Why we have chosen these methods, what did we want from the different 
  \subsection{Brainstorming}
  Short on brainstorming - general. \\
  Brainstorming is a technique/method where the participants toss around words or ideas that they have. In a typical brainstorming session there is a high tolerance and no idea or thought is discarded or neglected. Brainstorming can also be used frequently 
    \subsubsection{Description}
    We used brainstorming to come up with ideas when we decided our product in the beginning of the course. Later during the project we have used it to further develop ideas and requirements both for our own product and for the product we are developing. Our brainstorming sessions is like ordinary discussions and not always in a structured way.
    \subsubsection{Why chosen}

    Brainstorming is a good method to use in the beginning of a project to get started. It is a easy to use method that does not require a lot of preparations or materials. It is quite fast and the result is direct. 

    \subsubsection{Outcome}
    
    \subsection{Stakeholder analysis}
    Short of stakeholder analysis - general.\\
    This method is used to identify the people or organizations who are needed to ensure the success for the project. The important thing is to identify the stakeholders and to find their interests of the project. 

    \begin{itemize}
      \item Who are the stakeholders?
      \item What goals do they see for the system?
      \item Why would they like to contribute?
      \item What risks and costs do they see?
      \item What kind of solutions and suppliers do they see?
    \end{itemize}
    
    \subsubsection{Description}
    How we used it
    \subsubsection{Why chosen}
    Why we used it
    \subsubsection{Outcome}
    We came to the conclusion who it really is that we are making the system for. This led us to have a meeting with our costumers in which we realized that the government should be included, which is something that we hadn't discussed before.
    The context diagram helped us get an overview of the system and the different stakeholder we have.
    
    %vi kom fram till vem vi egentligen gjorde systemet för och det gjorde att vi pratade med kunden och då kom vi fram till att staten skulle vara med och det hade vi inte diskuterat tidigare. 
	%Contextdiagrammet hjälpte oss att få en överblick på systemet och de olika stakeholders vi har.
    
    \subsection{Similar companies}

    In order to get inspiration and see what there is today we looked at similar prducts from similar companies.
    The ones we looked at is

	\begin{itemize}
		\item Regional in Stockholm - SL
		\item Regional in Gothenburg - Västtrafik
		\item Reginoal in Skåne - Skånetrafiken
		\item Country trains - SJ
	\end{itemize}
    
    

    \subsubsection{Description}
    We looked at applications made for the above companies to see their solution on the same issue.
    \subsubsection{Why chosen}
    We used this method to get an idea of how a finished product might look like and to get inspiration to features to include in our solution.
    \subsubsection{Outcome}
    The method helped us to develop many of our system requirements and it generated questions that we could discuss with our costumers to get a better final product. 

    \subsection{Data model}
		
    \subsubsection{Description}
    How we used it
    \subsubsection{Why chosen}
    Why we used it
    \subsubsection{Outcome}

    \subsection{Context diagram}


    \subsubsection{Description}
    How we used it
    \subsubsection{Why chosen}
    Why we used it
    \subsubsection{Outcome}

  \section{Future work}
  %For next release

  In the future we want to use more elicitationstechniques to extract information from our costumers. And also to ease the work we are doing.
  Some of the techniques and methods we would like to incooperete in our work is the following:
  \begin{itemize}
  	\item Virtual Windows
  	\item Interview users
  	\item Questionnaires
  	\item Suppliers 
  \end{itemize}
 
  \section{Reflection}
  %What was successful/what was challenging
  %Example, what have we learned?
  %What could have been done differently?
  %What we have learned related to learning objectives

%  Jag ska översätta detta är tanken, Alex.

%  Det finns en anledning till varför eliciteringsbarriärer finns, det är inte lätt att ha kunder och att komma fram till en gemensam konsensus.
%  Vi har än så länge mest ägnat oss åt brainstorming och diskussioner mellan oss själva och med kunderna. Det är lätt att glömma de olika typerna av elicitering och att använda sig av dem. 

	There is a reason why the barriers of the elicitation exists, since it's not an easy task to generate a valid consensus together with the costumers.
	Mainly we have focused on brainstorming and open discussions among ourselves and the costumers. One should have in mind that the different types of elicitation is easily forgotten and sometimes difficult to know when to apply them.
  
 % Inte lönt att ta upp för mycket tekniska detaljer med kunderna.
 	During the startup time for the project it has been hard to divide the workload within the project group in an efficient manner, since we are eight persons working on pretty much the same tasks. This thing said, it's easy for some participants to be less contributing then others.
%	Än så länge har vi inte haft så stor möjlighet att dela upp arbetet och det är inte jätteeffektivt att sitta i en grupp på 8 personer och arbeta med samma sak, det är lätt att vissa inte blir lika delaktiga än andra. 



  \section{Personal statement}
  %Each member briefly explains its individual contributions to the project results
  
  \subsection{Alexander}
  \subsection{Emma}
	So far I have except for the common brainstorming, customer meetings and general work been working on a context diagram, data model and a first user task.	
  \subsection{Fredrik}
  	I have contributed to the data model and the elicitation technique similar companies. Except for that I have been involved in the brainstorming, customer meetings and common work. 
  \subsection{Hannes}
  \subsection{Johan}
  \subsection{Martin}
  \subsection{Mattias}
  \subsection{Philip}
  I have mostly been writing in this document and participated in discussions. 
  I've also handled a lot of the contact to and from our costumer group and the group we are cosutmers to.

\end{document}
