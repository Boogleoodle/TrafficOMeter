\documentclass[a4paper]{article}
\usepackage[utf8]{inputenc}
\usepackage[english]{babel}
\usepackage{hyperref}
\hypersetup{colorlinks=true, linkcolor=black, urlcolor=blue}

\title{Project Experiences version 2.0}
\author{Group D}
\date{\today}

\begin{document}
	\maketitle
	\thispagestyle{empty}
	\setcounter{page}{0}
	\pagebreak
	\tableofcontents
	\pagebreak
	
  \section{About this document}
  This document describes the process used, techniques chosen and why, learning outcomes and a personal statement from each team member on their individual contribution.
  \section{Elicitation Methods/Techniques}
  %Description of the techniques
  %Why we have chosen these methods, what did we want from the different 
  \subsection{Brainstorming}
Brainstorming is a technique where the participants toss around words or ideas that they have. In a typical brainstorming session there is a high tolerance and no idea or thought is discarded or neglected. Brainstorming can also be used frequently throughout the project.
    \subsubsection{Application}
We used brainstorming in the beginning of the project to put the idea in to a more concrete form. Our brainstorming sessions have been like ordinary discussions and not always in a structured way.
    \subsubsection{Why chosen}
Brainstorming is a good method to use in the beginning of a project to get started. It is a easy to use method that does not require a lot of preparations or materials. It is quite fast and the result is direct. Another reason we have used is that it does not require much preparation or documentation. Just a couple of people from the project group can brainstorm for 10 minutes on a topic and then share with the rest of the group what they came up with.

    \subsubsection{Outcome}
    Other than a more concrete idea, the brainstorming also yielded some requirements for the system, such as that it should consist of both an app and a widget. It also helped us to decide on the idea for the project in which we are clients, Boogle Calender.

    \subsection{Stakeholder analysis}
 This method is used to identify the people or organizations who are needed to ensure the success for the project. It is important not only to identify the stakeholders but also to find their interests and goals with the project. It can be of good help to have a set of questions you want answer to when you do the stakeholder analysis.

    \begin{itemize}
      \item Who are the stakeholders?
      \item What goals do they see for the system?
      \item Why would they like to contribute?
      \item What risks and costs do they see?
      \item What kind of solutions and suppliers do they see?
    \end{itemize}
    
    \subsubsection{Application}
 We combined brainstorming and stakeholder analysis to answer the questions above. This was done mostly within the group but we also discussed with our clients and with our supervisor.
    \subsubsection{Why chosen}
    Identifying the stakeholders in the beginning of the project will allow us to elicit more requirements early on. By knowing the stakeholders we can divide the requirements in to certain groups, which makes the specification and validation easier.
    \subsubsection{Outcome}
We came to the insight for whom we are really making the system. This led to a meeting with our costumers in which we realized that the government should be included, which is something that we hadn’t discussed before.
	The following stakeholders were identified:

    \begin{itemize}
      \item The commuters (the target users).
      \item The product owner. (Our clients)
      \item The regional public transportation administrators (Länstrafikbolagen). 
      \item The Swedish Government\footnote{Forgotten during stakeholder analysis but added during the construction of the Context diagram, see section \ref{subsec:context}}.
    \end{itemize}
    
    \subsection{Similar companies}
In order to get inspiration and see what similar services there are today, it can be a good idea to look at competitors' or partners' services. This is not a technique in itself but is often used with for example observation.
  
    \subsubsection{Application}
We decided to try the following services and combine it with an informal observation study where one person used the app and tried different features while another one observed. We discussed the individual alternatives during the experiment and compared them afterwards. The ones we looked at are:
\begin{itemize}
		\item SL app and widget (Regional in Stockholm)
		\item Västtrafik app and widget (Regional in Gothenburg)
		\item Skåetrafiken app and widget (Regional in Skåne)
		\item SJ mobile web page and app (Interregional trains Sweden)
	\end{itemize}
	
    \subsubsection{Why chosen}
We used this method to get an idea of how a finished product might look like
and to get inspiration to features to include in our solution.
    \subsubsection{Outcome}
The method helped us to develop many of our system requirements and it generated questions that we could discuss with our costumers to get a better final product.



%WORKING progress /Maritn
\subsection{Focus group \& design workshop}
    \subsubsection{Application}
We decided to combine a focus group with a design workshop as the topic was the same.
	
    \subsubsection{Why chosen}
We used this method in order to get input from potential users. It was important that the "PEOPLE" had not seen the project before and we therefore contacted people outside the course.
    \subsubsection{Outcome}
 
	
	
	\section{Specification Methods/Techniques}
    \subsection{Data model}
Data model is an abstract model of what data is needed in a system, not necessarily a requirement but in our case we use the model as a requirement.
	
    \subsubsection{Application}
    We chose the user as a starting point and tried to investigate what kind of data that we would need to be able to give the functionality required.
    \subsubsection{Why chosen}
    We used it to realize which kind of data the system would need.
    \subsubsection{Outcome}
	We got a lot of definitions out of putting it together since we needed to discuss what we meant by the entities in the model (coming in a data dictionary in release 2).

    \subsection{Virtual Window}
      A Virtual Window is basically a graphic screen with no actually functions or menus. It is used for visualizing the System's data so that the user and the developer can relate to it in a good manner. It can be used for developing the user interface.
      \subsubsection{Application}
        We used Virtual Widows to create a simple data model for the Application that the user interacts with the System. With the Virtual Windows and the Design workshop a suitable screen prototype could be made.
      \subsubsection{Why chosen}
        We chose to create some Virtual Windows because we wanted to know more of the data structure of the user part of the System.
      \subsubsection{Outcome}
        We got clearer view on how the system would interact with the costumer and what kind of data structure was needed.
    \subsection{Context diagram} \label{subsec:context}
	Picture of how the system fits in to/interacts with its inner domain. It is used to realize what systems/people directly interacts with the product.
    \subsubsection{Application}
    We chose the user as a starting point and reasoned about what the user needed to use the product for, from that we realized that the product needed to interact with a couple of systems. Then we reasoned about why the government wanted the product (to increase the use of public transport) which got us to realize that they would want statistics from the system.
    \subsubsection{Why chosen}
    We wanted to get an overview of how the system interacts with the actors and who might be affected by the system.
    \subsubsection{Outcome}
	When we were discussing the context diagram we realized that we were missing a stakeholder that we did not find during the stakeholder analysis. The missing stakeholder was the government that wants to get statistics from the system. 

	\subsection{QUPER}
		\textbf{quper är lite blandning av prio och spec, så var ska den ligga?}
		Used for quality requirements to deal with the problem of knowing if a little better for example performance would improve the product a lot or just be unnecessary. It is a way to see clearly in what range we want our quality requirements to be/how hard we want to work to satisfy them to a certain degree.
    	\subsubsection{Application}
    	We tried out QUPER on some of our quality requirements, for example how long time it takes from a press on the update button to the trips being shown.
    	\subsubsection{Why chosen}
    	QUPER was actually not chosen to be used for the system requirements, only parts were applied. When we tried to use it on some quality requirements we realised that we could not compare ourselves with other companies since to our knowledge there is no other company that shows the trips to multiple locations simultaneously. We tried to compare ourselves with Skånetrafiken but since they only show the trips to one destination we had difficulties scaling their results to an application with multiple destinations. The barriers also became a problem since for us it just became a guessing game of where to put them at it was not actually based on any knowledge of when there is a need for big structural reorganization. 
    	
    	The breakpoints on the other hand seems useful with or without the rest of the QUPER model so we have incorporated them on some requirements.

	    \subsubsection{Outcome}
		We would have liked to use the entire QUPER method on at least one of our quality requirements but because of the problems discussed above we had to settle for just specifying the breakpoints.		
		
		
	\subsection{Task Descriptions}
	    \subsubsection{Application}
    	\subsubsection{Why chosen}
    	\subsubsection{Outcome}


	\section{Prioritization Methods/Techniques}

  \subsection{100-dollar}
    \subsubsection{Application}
      The 100-dollar method was used to prioritize our requirements. 
    \subsubsection{Why chosen}
      This method is an easy way to get an overview over the rerquirements that is most important.
    \subsubsection{Outcome}
      From this method we were able to något and we also found some new requirements.

  \subsection{Insertion}
    \subsubsection{Application}
    \subsubsection{Why chosen}
    \subsubsection{Outcome}
    
  \subsection{Prioritization session with costumer}
    \subsubsection{Application}
      The costumer was asked to make a prioritization of the requirements we had so far, by ranking them from 1 to 5, where 1 was the highest priority and 5 was the lowest.
    \subsubsection{Why chosen}
      We wanted to get an idea of what our costumers thought was most important and to see if that was in line with our point of view.
    \subsubsection{Outcome}

  

  \section{Future work}
  %For next release
Before the next release we want to use more elicitation techniques to extract
information from our costumers. Some of the techniques and methods we would
like to incorporate in our work:

  \begin{itemize}
  	\item Virtual Windows
  	\item Interview users
  	\item Questionnaires
  	\item Suppliers 
  \end{itemize}
 
  \section{Reflection}
  %What was successful/what was challenging
  %Example, what have we learned?
  %What could have been done differently?
  %What we have learned related to learning objectives

While working with our project we have encountered several elicitation barriers. One that we noticed in particular is that the stakeholder cannot express what they need. We have also noticed that stakeholders can have conflicting views within themselves. Another barrier has been that our clients have suggested solutions rather than expressing their needs and demand.

An obstacle we have met is that we tend to forget to use several elicitation techniques. It is easy to get hung up on just brainstorming. Our group does not only have a big domain knowledge but also consists of potential end users. This gives us a false sense of security, as we think that we ourselves incorporate the different stakeholders' opinions and experiences. A solution to both these problems is to carry out different elicitation techniques and contact different stakeholders anyway. It is important to hear everyone's opinion, since we know from experience that more input will yield better results.

During the initial phase of the project it has been difficult to divide the workload. This is because we needed to be on the same page regarding several matters. However, trying to work all eight together has resulted inefficiency. This is something we are working with and we will try to divide the work into smaller independent tasks. We believe that this approach will result in higher efficiency.


  \section{Personal statements}
  %Each member briefly explains its individual contributions to the project results
  
  \subsection{Alexander}
    \subsubsection{R1}
    I have been mainly involved in the elicitation and the selection of elicitation techniques. I have also taken part of the group discussion, brainstorming and meetings with customers.
    \subsubsection{R2}
    \subsubsection{R3}
  
  \subsection{Emma}
    \subsubsection{R1}
    So far I have except for the common brainstorming, customer meetings and general work been working on a context diagram, data model and a first user task.
    \subsubsection{R2}
	In this release I have worked in particular with QUPER, quality goals and prioritization. I have also been involved in producing the validation checklist in cooperation with the rest of the group.
    \subsubsection{R3}
	
  \subsection{Fredrik}
    \subsubsection{R1}
    I have contributed to the data model and the elicitation technique similar companies. Except for that I have been involved in the brainstorming, customer meetings and common work.
    \subsubsection{R2}
    I have been participating in workshop with users and meetings with the customers. Done some more brainstorming, which resulted in a couple of new requirements. I've also been working on putting the requirements in the right place and corrected unambigous requirements.
    \subsubsection{R3}
  	
  \subsection{Hannes}
    \subsubsection{R1}
    I have helped producing the context diagram and planned for future quality requirements work. I have also gone to the meetings with stakeholders and been in the group discussions and brainstorming.
    \subsubsection{R2}
    \subsubsection{R3}
  
  \subsection{Johan}
    \subsubsection{R1}
    I have taken part in the elicitation process aswell as helping out with other tasks such as the context diagram. Further have I also been involved in the customer meetings and out internal brainstorming.
    \subsubsection{R2}
    \subsubsection{R3}
  
  \subsection{Martin}
    \subsubsection{R1}
    I have mostly coordinated the work, documenting the project experiences and trying different Gantt chart software. Have also had contact with our suppliers and participated in meetings with them and with our clients.
    \subsubsection{R2}
    \subsubsection{R3}
  
  \subsection{Mattias}
    \subsubsection{R1}
    I have been working with tools and setting up the document and file infrastructure. Other than this I have participated in the meetings.
    \subsubsection{R2}
	For this release I've mostly been working on the tasks and multiple refactorisations of the aforementioned tasks. Other than that I've been building a script to automatically generate all the external tex-files from scala files to make sure that the latest version of the models are in the system requirements document.
    \subsubsection{R3}
  \subsection{Philip}
    \subsubsection{R1}
  I have mostly been writing in this document and participated in discussions. I've also handled a lot of the contact to and from our costumer group and the group we are cosutmers to.
    \subsubsection{R2}
    Been working a bit with the design requirements and I have been working with putting requirements in the right section and I've been editing requirements to be easier to understand and more specified. 
    \subsubsection{R3}

\end{document}
