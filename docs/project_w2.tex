\documentclass[a4paper]{article}
\usepackage[utf8]{inputenc}

\title{Project Mission TraficOMeter v.2}
\author{Group D}
\date{\today}

\begin{document}
	\maketitle
	\thispagestyle{empty}
	\setcounter{page}{0}
	\pagebreak
	\tableofcontents
	\pagebreak

	\section{Bakgrund} % (fold)
	\label{sec:background}
		Alla som någon gång rest med kollektivtrafik vet hur tidskrävande det kan vara att söka upp
snabbaste resväg. Vi vill underlätta denna process genom att skapa ett system som automatiskt
tar fram de snabbaste reseförslagen till användarens förinställda destinationer.
	
	% section background (end)

	\section{Mål} % (fold)
	\label{sec:m_l}
		Målet med projektet är att utveckla en app/widget där användaren kan se den snabbaste resan till användarens sparade destinationer. Resealternativen anpassas beroende på var användaren befinner sig, så att startpunkt alltid tas från användarens geografiska position.

		% funktionallitet
		\begin{itemize}
			\item Systemet ska stödja både Android- och iOS-enheter.
			\item Det ska gå att söka upp destinationer och spara dem.
			\item Systemet ska kunna hantera flera sparade destinationer.
			\item Systemet ska använda användarens geografiska position som startpunkt för reseförslagen.
			\item Systemet ska stödja all sorts kollektivtrafik i Sverige.
			\item Användaren ska kunna specificera vilka typer av färdmedel som ska användas.
			\item Systemet ska automatiskt visa det snabbaste resealternativet.
			\item Det ska gå att komma åt sparade destinationer från olika enheter.
			\item Användaren ska kunna se tiden till sina sparade resmål på sin hemmaskärm.
		\end{itemize}

	% section m_l (end)

	\section{Deltagare} % (fold)
	\label{sec:deltagare}
		\subsection{Produktägare}
			\begin{itemize}
				\item[SCCVM] Aleksandar Zezovski, dat11aze
				\item[DRM] Oscar Hinton, dat11ohi
				\item[P3RM] Cornelia Jeppsson, dat11cje
				\item[QRM] Hanna Hultgren, ada10hhu
				\item[EPM] Johan Brantberg, ada10jb1
				\item[TDEVM] Troung Hoang, ada09tho
			\end{itemize}
		
		\subsection{Utvecklare}
			\begin{itemize}
				\item[P3RM]	Martin Richter 
				\item[SCCVM]	Philip Holgersson
				\item[TDEVM]	Fredrik Åkerberg \& Mattias Eklund
				\item[EPM]	Alexander Nässlander \& Johan Mattsson
				\item[QRM]	Hannes Johansson
				\item[DRM]	Emma Holmberg Ohlsson
			\end{itemize}
		
	% section deltagare (end)

	\section{Stakeholders} % (fold)
	\label{sec:stakeholders}
		\begin{itemize}
			\item Produktägarna
			\item Framtida användare av systemet
			\item Länens transportsystem för kollektivtrafik
		\end{itemize}
	% section stakeholders (end)

	\section{Planerade aktiviteter}
		%Står att vi ska ha en beskrivning av detta.. Kanske typ att vi ska ha eliciterings möte, och validering och sånt?.
		Vi ska elicitera, validera och prioritera krav.\\
		
		Vi har börjat med våran elicitering då vi påbörjat en stakeholder analysis.
		
		Vi funderar på att använda enkäter för att få information om/från användarna. Utfrågning av leverantärer av lokal kollektivtrafik.
		
		Produktägarna
			prototyping
			design workshop
			domain ws
			goal/domain anaysis

	\section{Leverabler} % (fold)
	\label{sec:deliverables}
		Vi plannerar att släppa följande leverabler:
		\begin{itemize}
			\item Systemkrav
			\item Projekterfarenheter
			\item Valideringsrapport
			\item Valideringschecklista
		\end{itemize}
		Systemkrav och projekterfarenheter version 1 släpps den 24/11.\\
		Systemkrav och projekterfarenheter version 2 samt valideringsrapport släpps den 8/12.\\
		Valideringschecklistan släpps den 12/12.\\
		Systemkrav och projekterfarenheter version 3 släpps den 21/12\\ 		
	% section deliverables (end)

	\section{Tidsuppskattning}
		%vi ska ha någon form av diagram över planerad tid per vecka och person?	
	
\end{document}
