\documentclass[a4paper]{article}
\usepackage[utf8]{inputenc}

\title{Project Mission TraficOMeter v.2}
\author{Group D}
\date{\today}

\begin{document}
	\maketitle
	\thispagestyle{empty}
	\setcounter{page}{0}
	\pagebreak
	\tableofcontents
	\pagebreak

	\section{Bakgrund} % (fold)
	\label{sec:background}
		Den nya regeringen (to be) har beslutat att man vill sänka växthusgaserna genom att fler människor ska börja resa kollektivt. De har därför anlitat produktägarna att utveckla ett system som gör det lättare för kollektivtrafikanter att resa. Målet ska uppnås genom att binda ihop Sveriges länstrafikbolag och hjälpa användaren hitta den snabbaste vägen.
	
	% section background (end)

	\section{Mål} % (fold)
	\label{sec:m_l}
		Målet med vårat projekt är att utveckla en app/widget där användaren kan se den snabbaste resan till användarens sparade destinationer. Resealternativen anpassas beroende på var användaren befinner sig, så att startpunkt alltid tas från användarens geografiska position.

		% funktionallitet
		\begin{itemize}
			\item Systemet ska stödja både Android- och iOS-enheter.
			\item Det ska gå att söka upp destinationer och spara dem.
			\item Systemet ska kunna hantera flera sparade destinationer.
			\item Systemet ska använda användarens geografiska position som startpunkt för reseförslagen.
			\item Systemet ska stödja buss, tåg och färja i Sverige.
			\item Användaren ska kunna specificera vilka typer av färdmedel som ska användas.
			\item Systemet ska automatiskt visa det snabbaste resealternativet.
			\item Det ska gå att komma åt sparade destinationer från olika enheter.
			\item Användaren ska kunna se tiden till sina sparade resmål på sin hemmaskärm.
		\end{itemize}

	% section m_l (end)

	\section{Deltagare} % (fold)
	\label{sec:deltagare}
		\subsection{Produktägare}
			\begin{itemize}
				\item[SCCVM] Aleksandar Zezovski, dat11aze
				\item[DRM] Oscar Hinton, dat11ohi
				\item[P3RM] Cornelia Jeppsson, dat11cje
				\item[QRM] Hanna Hultgren, ada10hhu
				\item[EPM] Johan Brantberg, ada10jb1
				\item[TDEVM] Troung Hoang, ada09tho
			\end{itemize}
		
		\subsection{Utvecklare}
			\begin{itemize}
				\item[P3RM]	Martin Richter 
				\item[SCCVM]	Philip Holgersson
				\item[TDEVM]	Fredrik Åkerberg \& Mattias Eklund
				\item[EPM]	Alexander Nässlander \& Johan Mattsson
				\item[QRM]	Hannes Johansson
				\item[DRM]	Emma Holmberg Ohlsson
			\end{itemize}
		
	% section deltagare (end)

	\section{Stakeholders} % (fold)
	\label{sec:stakeholders}
		\begin{itemize}
			\item Produktägarna
			\item Framtida användare av systemet
			\item Länens transportsystem för kollektivtrafik
		\end{itemize}
	% section stakeholders (end)

	\section{Planerade aktiviteter}
		%Står att vi ska ha en beskrivning av detta.. Kanske typ att vi ska ha eliciterings möte, och validering och sånt?.
		Vi ska elicitera, validera och prioritera kraven på systemet.\\
		
		Vi har i nuläget påbörjat vår elicitering genom att göra en stakeholder analysis.
		För att få den information vi behöver från våra stakeholders så funderar vi på att använda enkäter för slutanvändare och utfrågning av leverantörer när det gäller länens kollektivtrafik. För att elicitera krav från produktägarna kommmer vi bland annat att använda prototyping, design workshops, domain workshops och goal-domain analysis.

	\section{Leverabler} % (fold)
	\label{sec:deliverables}
		Vi planerar att släppa följande leverabler:
		\begin{itemize}
			\item Systemkrav
			\item Projekterfarenheter
			\item Valideringsrapport
			\item Valideringschecklista
		\end{itemize}
		Systemkrav och projekterfarenheter version 1 släpps den 24/11.\\
		Systemkrav och projekterfarenheter version 2 samt valideringsrapport släpps den 8/12.\\
		Valideringschecklistan släpps den 12/12.\\
		Systemkrav och projekterfarenheter version 3 släpps den 21/12\\ 		
	% section deliverables (end)

	\section{Tidsuppskattning}
		%vi ska ha någon form av diagram över planerad tid per vecka och person?	
		Per vecka så räknar vi som team med att lägga ner i genomsnitt 10 timmar per person. Denna tid kommer med stor sannolikhet att fördela sig så att det är mer än 10 timmar veckan innan en release och något mindre precis efter en release. 	
\end{document}
