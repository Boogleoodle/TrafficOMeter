\documentclass[a4paper]{article}

\usepackage[T1]{fontenc}
\usepackage[english]{babel}
\usepackage{graphicx}
\usepackage[nottoc]{tocbibind}
\usepackage[utf8]{inputenc}

\begin{document}
	\thispagestyle{empty}
	\setcounter{page}{0}
	\pagebreak
	\tableofcontents
	\pagebreak

\section{Allmänt}
\begin{itemize}

	\item DONE: Ska vi kalla det locations eller destination det som sparas, tror vi har olika lite överallt
	\item samma med alla former av resa, vi är inte konsekventa med ordvalet
	\item DONE: Lägga till prio från fler stakeholders på kraven
	\item \textbf{DONE: identifiers för krav är inte unika}
	\item ord som används i förklaringarna av context diagrammet förklars inte förän vi data modelen senare.. göra om detta på vettigt sätt
\end{itemize}


\section{features}
	\subsection{searchdestination}
		\begin{itemize}
			\item gör denna tydligare. 
			\item är det bara för att man ska kunna lägga till genom att söka efter den?, och isf har vi något annat krav för nån annan möjlighet eller är detta enda sättet att lägga till? har för mig att vi har ett till, kan man inte gruppera dem tillsammans på något sätt
		\end{itemize}
		

		
	\subsection{orderdestinations}
		\begin{itemize}
			\item Här antar vi att vi har en widget, så antingen göra krav på det eller skriva om detta krav
			\item här är example ett exempel och inte förslag på lösning, välj hur vi ska ha det och gör det konsekvent
 		
		\end{itemize}
		
	\subsection{extractstatistics}
vi får antingen specificera vilken typ av statistik  här eller bygga på med  fler specifika som vi redan har några. nu är det skumt med ett övergripande och några specifika


	DONE		
	\subsection{getroutestatistics}
	detta borde förtydligas och vi borde ha ett för de resor som man faktiskt har klickat upp för att man vill ta dem så att staten kan se vad som ör interessant när man är på en speciell plats. Tycker att det ska vara ett separat.
	
	DONE

	\subsection{getdatafromproviders}
	ska det vara att producten ska göra detta?
	

\end{document}