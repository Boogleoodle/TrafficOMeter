\documentclass[a4paper]{article}
\usepackage[utf8]{inputenc}
\usepackage[english]{babel}

\begin{document}
Egna tankar

Inte lönt att ta upp för mycket tekniska detaljer med kunderna.

det finns en anledning till varför eliciteringsbarriärer finns, det är inte lätt att ha kunder och att komma fram till en gemensam konsensus.

Vi har än så länge mest ägnat oss åt brainstorming och diskussioner mellan oss själva och med kunderna. Det är lätt att glömma de olika typerna av elicitering och att använda sig av dem. 

Än så länge har vi inte haft så stor möjlighet att dela upp arbetet och det är inte jätteeffektivt att sitta i en grupp på 8 personer och arbeta med samma sak, det är lätt att vissa inte blir lika delaktiga än andra. 

Text att lämna in - typ

vi har gjort stakeholderanalys och contextdiagram - vi kom fram till vem vi egentligen gjorde systemet för och det gjorde att vi pratade med kunden och då kom vi fram till att staten skulle vara med och det hade vi inte diskuterat tidigare. Contextdiagrammet hjälpte oss att få en överblick på systemet och de olika stakeholders vi har. 

	
\end{document}
