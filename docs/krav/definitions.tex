

\begin{lstlisting}
Term MeansOfTransportation
	Spec Type of vehicle in which the public transport takes place
	Example Bus, train, or ferry, or a combination of them
Term User
	Spec An individual who uses the system
	Example An individual that wants to travel between pre-set destinations
Term Destination
	Spec A geographical location that has been saved and given a name by the user.
	Example Destination A = Name: Home, lat: 55.7112325, lon: 13.1867275. Destination B = Name: School, lat: 55.71037075564825, lon: 13.208270072937012
Term Time
	Spec Time of departure
	Example Means of transportation A departs at time B
Term LineOfTransportation
	Spec A set of 'stop's that is served by the same vehicle. There is one line for each direction between the stops.
	Example Bus 6 in Lund towards S:t Lars, Bus 171 between Malmoe and Lund
Term GeographicalLocation
	Spec Consists of GPS coordinates to determine a location
	Example lat: 55.71037075564825, lon: 13.208270072937012
Term Trip
	Spec A specific travel between two geographical (GPS) location
	Example Between a users current position and one of its destinations
Term Stop
	Spec Place where a public transportation vehicle stop to pick up and let off passengers.
	Example Bus A stops at stop X to let set of persons P on and set of persons Q off.
Term Route
	Spec The path between two different stops. Consists of one or more 'sub-line's
	Example From bus stop X to bus stop Z, (change at bus stop Y e.g X->Y, Y->Z)
Term Timetable
	Spec The predefined departure times set by the regional public transport office for a specific means of transportation for every line of transportation
	Example Bus A departs at time X hours and Y minutes direction Z
Term Sub-line
	Spec A subset of a line of transportation that is part of a bigger 'route'
	Example Bus 3, from Lund C to Universitetssjukhuset

\end{lstlisting}