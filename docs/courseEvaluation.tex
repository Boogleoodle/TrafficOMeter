\documentclass[a4paper]{article}
\usepackage[utf8]{inputenc}
\usepackage[english]{babel}
\usepackage{hyperref}
\usepackage{hhline}
\hypersetup{colorlinks=true, linkcolor=black, urlcolor=blue}
\usepackage{fancyhdr}


\title{Course Evaluation ETS170 Version 1.0}
\author{Group D\\ Mattias Eklund, Philip Holgersson, Emma Holmberg Ohlsson,\\ Hannes Johansson, Johan Mattsson, Alexander Nässlander\\Martin Richter, Fredrik Åkerberg}
\date{\today}

\begin{document}
	\maketitle
	\thispagestyle{empty}
	\pagestyle{empty}
	\newpage
	\clearpage
	\tableofcontents
	\newpage
	\setcounter{page}{1}

%=============================================================
%Headers and footers
%=============================================================
\pagestyle{fancy}
\lhead{Traffic-O-Meter}
\chead{Course Evaluation ETS170}
\rhead{Version 1.0} %\thepage}
%\cfoot{\thepage}
\renewcommand{\headrulewidth}{0.4pt}
\renewcommand{\footrulewidth}{0.4pt}

%=============================================================
%Start of document
%=============================================================

	
	\section{About this document}
	In this document we evaluate the course and give recommendations for improvement.
	
	\section{Lectures}
	The lectures gave a good overview of the course but for some concepts we were only told that they exist and did not get a lot of information. It might possibly be better to have a slide with important words in the concepts and go into explaining some words a little bit more instead than giving almost no explanation at all.

	\section{Exercises}
	The exercises were a great opportunity for the group to work on new parts for the project. It was really helpful to be able to tackle new concepts with the help of an assistant so that all initial questions could be answered immediately and more progress could be made.
	It was good that part of the exercises were about using the new concepts from the lectures before work on the project could be done.
	Something that could be improved is that some assignments were only looking through the book, were it might be more beneficial to write all types of requirements instead?.

	\section{Labs}
	The labs were well structured so that the assignments were clear. The first lab about reqT didn't really relate to the course as much as we would have liked. One solution to this is to change the course content to include something about tools or to give this part less attention. The second lab was much more course related and really gave us some practice on some of the parts of the course.

	\section{Project}
	It is quite hard to work efficiently in an eight person group, and therefore we are not sure that it was a great choice to increase the group size. On the other hand the work load would have been even greater if we had been a smaller group.
	
	The project has been quite time consuming and it might be a good idea to restrict the scope of the projects even more. Other than the work load the project has been great, it has really given us a reason to learn about all parts of requirements engineering.

	\section{Exam Questions}
	The writing of exam questions gave us a reason to read some of the course material. The points awarded for the exam questions weren't really clear why some groups got more than others since the comments on the questions seemed about the same level of good/bad.	
	
	\section{Presentation}
The presentations so they were quite short so in general they were very similar to	one and another. A good idea could be to divide the content so that one group presents their experiences, one their elicitation etc.

	\section{Google groups}
	The group mail was very helpful since we got answers to problems we might also encounter.

	\section{reqT}
	Some possibilities to affect the output of the models when exporting to latex would have been nice. For example an empty line between all requirements would have given a better output.

	When using words like User, is and has in the text they become highlighted. I would be good to make some changes in ReqT that words that are used by ReqT can also be used as ordinary text.


	\section{Other comments}
	All information about the project should be updated to reflect the new group size (for example the course program has not been updated).
\end{document}


%Hannes
%Hannes
%Hannes
%Hannes
%Hannes
%Hannes
%Hannes
%Hannes
%Hannes
%Hannes
%Hannes
%Hannes
%Hannes
%Hannes
%Hannes
%Hannes
%Hannes
%Hannes
%Hannes
%Hannes
%Hannes
%Hannes
%Hannes
%Hannes
%Hannes
%Hannes
%Hannes
%Hannes
%Hannes
%Hannes
%Hannes
%Hannes
%Hannes
%Hannes
%Hannes
%Hannes
%Hannes
%Hannes
%Hannes
%Hannes
%Hannes
%Hannes
%Hannes
%Hannes
%Hannes
%Hannes
%Hannes
%Hannes
%Hannes
%Hannes
%Hannes
%Hannes
%Hannes
%Hannes
%Hannes
%Hannes
%Hannes
%Hannes
%Hannes
%Hannes
%Hannes
%Hannes
%Hannes
%Hannes
%Hannes
%Hannes
%Hannes
%Hannes
%Hannes
%Hannes
%Hannes
%Hannes
%Hannes
%Hannes
%Hannes
%Hannes
%Hannes
%Hannes
%Hannes
%Hannes
%Hannes
%Hannes
%Hannes
%Hannes
%Hannes
%Hannes
%Hannes
%Hannes
%Hannes
%Hannes
%Hannes
%Hannes
%Hannes
%Hannes
%Hannes
%Hannes
%Hannes
%Hannes
%Hannes
%Hannes
%Hannes
%Hannes
%Hannes
%Hannes
%Hannes
%Hannes
%Hannes
%Hannes
%Hannes
%Hannes
%Hannes
%Hannes
%Hannes
%Hannes
%Hannes
%Hannes
%Hannes
%Hannes
%Hannes
%Hannes
%Hannes
%Hannes
%Hannes
%Hannes
%Hannes
%Hannes
%Hannes
%Hannes
%Hannes
%Hannes
%Hannes
%Hannes
%Hannes
%Hannes
%Hannes
%Hannes
%Hannes
%Hannes
%Hannes
%Hannes
%Hannes
%Hannes
%Hannes
%Hannes
%Hannes
%Hannes
%Hannes
%Hannes
%Hannes
%Hannes
%Hannes
%Hannes
%Hannes
%Hannes
%Hannes
%Hannes
%Hannes
%Hannes
%Hannes
%Hannes
%Hannes
%Hannes
%Hannes
%Hannes
%Hannes
%Hannes
%Hannes
%Hannes
%Hannes
%Hannes
%Hannes
%Hannes
%Hannes
%Hannes
%Hannes
%Hannes
%Hannes
%Hannes
%Hannes
%Hannes
%Hannes
%Hannes
%Hannes
%Hannes
%Hannes
%Hannes
%Hannes
%Hannes
%Hannes
%Hannes
%Hannes
%Hannes
%Hannes
%Hannes
%Hannes
%Hannes
%Hannes
%Hannes
%Hannes
%Hannes
%Hannes
%Hannes
%Hannes
%Hannes
%Hannes
%Hannes
%Hannes
%Hannes
%Hannes
%Hannes
%Hannes
%Hannes
%Hannes
%Hannes
%Hannes
%Hannes
%Hannes
%Hannes
%Hannes
%Hannes
%Hannes
%Hannes
%Hannes
%Hannes
%Hannes
%Hannes
%Hannes
%Hannes
%Hannes
%Hannes
%Hannes
%Hannes
%Hannes
%Hannes
%Hannes
%Hannes
%Hannes
%Hannes
%Hannes
%Hannes
%Hannes
%Hannes
%Hannes
%Hannes
%Hannes
%Hannes
%Hannes
%Hannes
%Hannes
%Hannes
%Hannes
%Hannes
%Hannes
%Hannes
%Hannes
%Hannes
%Hannes
%Hannes
%Hannes
%Hannes
%Hannes
%Hannes
%Hannes
%Hannes
%Hannes
%Hannes
%Hannes
%Hannes
%Hannes
%Hannes
%Hannes
%Hannes
%Hannes
%Hannes
%Hannes
%Hannes
%Hannes
%Hannes
%Hannes
%Hannes
%Hannes
%Hannes
%Hannes
%Hannes
%Hannes
%Hannes
%Hannes
%Hannes
%Hannes
%Hannes
%Hannes
%Hannes
%Hannes
%Hannes
%Hannes
%Hannes
%Hannes
%Hannes
%Hannes
%Hannes
%Hannes
%Hannes
%Hannes
%Hannes
%Hannes
%Hannes
%Hannes
%Hannes
%Hannes
%Hannes
%Hannes
%Hannes
%Hannes
%Hannes
%Hannes
%Hannes
%Hannes
%Hannes
%Hannes
%Hannes
%Hannes
%Hannes
%Hannes
%Hannes
%Hannes
%Hannes
%Hannes
%Hannes
%Hannes
%Hannes
%Hannes
%Hannes
%Hannes
%Hannes
%Hannes
%Hannes
%Hannes
%Hannes
%Hannes
%Hannes
%Hannes
%Hannes
%Hannes
%Hannes
%Hannes
%Hannes
%Hannes
%Hannes
%Hannes
%Hannes
%Hannes
%Hannes
%Hannes
%Hannes
%Hannes
%Hannes
%Hannes
%Hannes
%Hannes
%Hannes
%Hannes
%Hannes
%Hannes
%Hannes
%Hannes
%Hannes
%Hannes
%Hannes
%Hannes
%Hannes
%Hannes
%Hannes
%Hannes
%Hannes
%Hannes
%Hannes
%Hannes
%Hannes
%Hannes
%Hannes
%Hannes
%Hannes
%Hannes
%Hannes
%Hannes
%Hannes
%Hannes
%Hannes
%Hannes
%Hannes
%Hannes
%Hannes
%Hannes
%Hannes
%Hannes
%Hannes
%Hannes
%Hannes
%Hannes
%Hannes
%Hannes
%Hannes
%Hannes
%Hannes
%Hannes
%Hannes
%Hannes
%Hannes
%Hannes
%Hannes
%Hannes
%Hannes
%Hannes
%Hannes
%Hannes
%Hannes
%Hannes
%Hannes
%Hannes
%Hannes
%Hannes
%Hannes
%Hannes
%Hannes
%Hannes
%Hannes
%Hannes
%Hannes
%Hannes
%Hannes
%Hannes
%Hannes
%Hannes
%Hannes
%Hannes
%Hannes
%Hannes
%Hannes
%Hannes
%Hannes
%Hannes
%Hannes
%Hannes
%Hannes
%Hannes
%Hannes
%Hannes
%Hannes
%Hannes
%Hannes
%Hannes
%Hannes
%Hannes
%Hannes
%Hannes
%Hannes
%Hannes
%Hannes
%Hannes
%Hannes
%Hannes
%Hannes
%Hannes
%Hannes
%Hannes
%Hannes
%Hannes
%Hannes
%Hannes
%Hannes
%Hannes
%Hannes
%Hannes
%Hannes
%Hannes
%Hannes
%Hannes
%Hannes
%Hannes
%Hannes
%Hannes
%Hannes
%Hannes
%Hannes
%Hannes
%Hannes
%Hannes
%Hannes
%Hannes
%Hannes
%Hannes
%Hannes
%Hannes
%Hannes
%Hannes
%Hannes
%Hannes
%Hannes
%Hannes
%Hannes
%Hannes
%Hannes
%Hannes
%Hannes
%Hannes
%Hannes
%Hannes
%Hannes
%Hannes
%Hannes
%Hannes
%Hannes
%Hannes
%Hannes
%Hannes
%Hannes
%Hannes
%Hannes
%Hannes
%Hannes
%Hannes
%Hannes
%Hannes
%Hannes
%Hannes
%Hannes
%Hannes
%Hannes
%Hannes
%Hannes
%Hannes
%Hannes
%Hannes
%Hannes
%Hannes
%Hannes
%Hannes
%Hannes
%Hannes
%Hannes
%Hannes
%Hannes
%Hannes
%Hannes
%Hannes
%Hannes
%Hannes
%Hannes
%Hannes
%Hannes
%Hannes
%Hannes
%Hannes
%Hannes
%Hannes
%Hannes
%Hannes
%Hannes
%Hannes
%Hannes
%Hannes
%Hannes
%Hannes
%Hannes
%Hannes
%Hannes
%Hannes
%Hannes
%Hannes
%Hannes
%Hannes
%Hannes
%Hannes
%Hannes
%Hannes
%Hannes
%Hannes
%Hannes
%Hannes
%Hannes
%Hannes
%Hannes
%Hannes
%Hannes
%Hannes
%Hannes
%Hannes
%Hannes
%Hannes
%Hannes
%Hannes
%Hannes
%Hannes
%Hannes
%Hannes
%Hannes
%Hannes
%Hannes
%Hannes
%Hannes
%Hannes
%Hannes
%Hannes
%Hannes
%Hannes
%Hannes
%Hannes
%Hannes
%Hannes
%Hannes
%Hannes
%Hannes
%Hannes
%Hannes
%Hannes
%Hannes
%Hannes
%Hannes
%Hannes
%Hannes
%Hannes
%Hannes
%Hannes
%Hannes
%Hannes
%Hannes
%Hannes
%Hannes
%Hannes
%Hannes
%Hannes
%Hannes
%Hannes
%Hannes
%Hannes
%Hannes
%Hannes
%Hannes
%Hannes
%Hannes
%Hannes
%Hannes
%Hannes
%Hannes
%Hannes
%Hannes
%Hannes
%Hannes
%Hannes
%Hannes
%Hannes
%Hannes
%Hannes
%Hannes
%Hannes
%Hannes
%Hannes
%Hannes
%Hannes
%Hannes
%Hannes
%Hannes
%Hannes
%Hannes
%Hannes
%Hannes
%Hannes
%Hannes
%Hannes
%Hannes
%Hannes
%Hannes
%Hannes
%Hannes
%Hannes
%Hannes
%Hannes
%Hannes
%Hannes
%Hannes
%Hannes
%Hannes
%Hannes
%Hannes
%Hannes
%Hannes
%Hannes
%Hannes
%Hannes
%Hannes
%Hannes
%Hannes
%Hannes
%Hannes
%Hannes
%Hannes
%Hannes
%Hannes
%Hannes
%Hannes
%Hannes
%Hannes
%Hannes
%Hannes
%Hannes
%Hannes
%Hannes
%Hannes
%Hannes
%Hannes
%Hannes
%Hannes
%Hannes
%Hannes
%Hannes
%Hannes
%Hannes
%Hannes
%Hannes
%Hannes
%Hannes
%Hannes
%Hannes
%Hannes
%Hannes
%Hannes
%Hannes
%Hannes
%Hannes
%Hannes
%Hannes
%Hannes
%Hannes
%Hannes
%Hannes
%Hannes
%Hannes
%Hannes
%Hannes
%Hannes
%Hannes
%Hannes
%Hannes
%Hannes
%Hannes
%Hannes
%Hannes
%Hannes
%Hannes
%Hannes
%Hannes
%Hannes
%Hannes
%Hannes
%Hannes
%Hannes
%Hannes
%Hannes
%Hannes
%Hannes
%Hannes
%Hannes
%Hannes
%Hannes
%Hannes
%Hannes
%Hannes
%Hannes
%Hannes
%Hannes
%Hannes
%Hannes
%Hannes
%Hannes
%Hannes
%Hannes
%Hannes
%Hannes
%Hannes
%Hannes
%Hannes
%Hannes
%Hannes
%Hannes
%Hannes
%Hannes
%Hannes
%Hannes
%Hannes
%Hannes
%Hannes
%Hannes
%Hannes
%Hannes
%Hannes
%Hannes
%Hannes
%Hannes
%Hannes
%Hannes
%Hannes
%Hannes
%Hannes
%Hannes
%Hannes
%Hannes
%Hannes
%Hannes
%Hannes
%Hannes
%Hannes
%Hannes
%Hannes
%Hannes
%Hannes
%Hannes
%Hannes
%Hannes
%Hannes
%Hannes
%Hannes
%Hannes
%Hannes
%Hannes
%Hannes
%Hannes
%Hannes
%Hannes
%Hannes
%Hannes
%Hannes
%Hannes
%Hannes
%Hannes
%Hannes
%Hannes
%Hannes
%Hannes
%Hannes
%Hannes
%Hannes
%Hannes
%Hannes
%Hannes
%Hannes
%Hannes
%Hannes
%Hannes
%Hannes
%Hannes
%Hannes
%Hannes
%Hannes
%Hannes
%Hannes
%Hannes
%Hannes
%Hannes
%Hannes
%Hannes
%Hannes
%Hannes
%Hannes
%Hannes
%Hannes
%Hannes
%Hannes
%Hannes
%Hannes
%Hannes
%Hannes
%Hannes
%Hannes
%Hannes
%Hannes
%Hannes
%Hannes
%Hannes
%Hannes
%Hannes
%Hannes
%Hannes
%Hannes
%Hannes
%Hannes
%Hannes
%Hannes
%Hannes
%Hannes
%Hannes
%Hannes
%Hannes
%Hannes
%Hannes
%Hannes
%Hannes
%Hannes
%Hannes
%Hannes
%Hannes
%Hannes
%Hannes
%Hannes
%Hannes
%Hannes
%Hannes
%Hannes
%Hannes
%Hannes
%Hannes
%Hannes
%Hannes
%Hannes
%Hannes
%Hannes
%Hannes
%Hannes
%Hannes
%Hannes
%Hannes
%Hannes
%Hannes
%Hannes
%Hannes
%Hannes
%Hannes
%Hannes
%Hannes
%Hannes
%Hannes
%Hannes
%Hannes
%Hannes
%Hannes
%Hannes
%Hannes
%Hannes
%Hannes
%Hannes
%Hannes
%Hannes
%Hannes
%Hannes
%Hannes
%Hannes
%Hannes
%Hannes
%Hannes
%Hannes
%Hannes
%Hannes
%Hannes
%Hannes
%Hannes
%Hannes
%Hannes
%Hannes
%Hannes
%Hannes
%Hannes
%Hannes
%Hannes
%Hannes
%Hannes
%Hannes
%Hannes
%Hannes
%Hannes
%Hannes
%Hannes
%Hannes
%Hannes