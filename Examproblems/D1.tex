\documentclass[a4paper]{article}
\usepackage[utf8]{inputenc}
\usepackage[english]{babel}
\title{Exam Questions 1}
\author{Group D\\ Emma Holmberg Ohlsson, Alexander Nässlander,\\Martin Richter, Hannes Johansson, Johan Mattsson,\\Phillip Holgersson, Mattias Eklund, Fredrik Åkerberg}
\date{\today}

\begin{document}
	\maketitle
	\thispagestyle{empty}
	\setcounter{page}{0}
	\pagebreak

\section{Context Diagram}
\subsection*{Proposition}
A context diagram is only useful in a tender situation.
\subsection*{Reason}
In a tender situation functional and design requirements are of none or little use. A context diagram describes how  the different part of the system (inner domain) is connected.
\subsection*{Correct Answer}
D - Proposition is false, but the reason is a true statement.
\subsection*{Motivation}
Context diagrams can be useful in all sorts of projects to describe the domain, and not just in tender situations. If we specified design and functional requirements we would rule out possible suppliers before we had even started.
\subsection*{Reference}
Lau: Chapter 1 pages 20
\subsection*{Learning objective}
1.1.1, 1.1.2, 1.1.3
\subsection*{Main responsible}
Emma Holmberg Ohlsson



\section{Business Goals}
\subsection*{Proposition}
Business goals are used for the supplier to understand the domain better. They can be used to verify that the goal requirements are taken into account during development.
\subsection*{Reason}
Business goals are requirements that are defined in the requirement specification so that they can be verified after the project is done and the system is implemented.
\subsection*{Correct Answer}
C - The proposition is true, but the reason is false.
\subsection*{Motivation}
Business goals should not be used as requirements. A supplier can not be held responsible for a system to fulfill the business goals for a system. They are used so that the supplier understands the domain and the reason behind the system better
\subsection*{Reference}
Lau: Chapter 1 page 30
\subsection*{Learning objective}
1.1.1, 1.1.2
\subsection*{Main responsible}
Martin Richter


\section{Data expressions}
\subsection*{Proposition}
Data expressions give a good overview for the customer.
\subsection*{Reason}
Data expressions are easier to understand for customers than E/R-diagrams. 
\subsection*{Correct Answer}
D - The proposition is false, but the reason is a true statement.
\subsection*{Motivation}
It can be hard for people without an IT-background to understand E/R-diagrams. Data expressions are easier to understand on a detailed level for the customer but gives no overview. 
\subsection*{Reference}
Lau: Chapter 2 pages 60, 63-64
\subsection*{Learning objective}
1.1.1, 1.1.2
\subsection*{Main responsible}
Alexander Nässlander


\section{Data requirements}
\subsection*{Proposition}
A data model and a data dictionary both need to exist if one of them does.
\subsection*{Reason}
A data model describes what entities exist and their relations. A data dictionary describes the entities.
\subsection*{Correct Answer}
D - Proposition is false, but the reason is a true statement.
\subsection*{Motivation}
Data model should have an accompanying data dictionary to explain it, but a data dictionary can exist on its own.
\subsection*{Reference}
Lau: Chapter 2 pages 44-60
\subsection*{Learning objective}
1.1.1, 1.1.3
\subsection*{Main responsible}
Emma Holmberg Ohlsson


\section{Vivid Scenarios}
\subsection*{Proposition}
Vivid scenarios should not be used as requirements
\subsection*{Reason}
Vivid scenarios are too hard to verify to be good requirements, and they do not cover all task that the system should support.
\subsection*{Correct Answer}
A - Both the proposition and the reason are correct statements, AND the reason explains the proposition in a correct way.
\subsection*{Motivation}
Vivid scenarios does not cover all task that should be performed by the system, and it is hard to verify that they are fulfilled in the system during development. Therefore they are not suitable as requirements. On the other hand they can be useful to include in requirement documentation to give the developers (and the customers) a sense of how the system will be used.
\subsection*{Reference}
Lau: Chapter 3 pages 114-115
\subsection*{Learning objective}
1.1.1, 1.1.3
\subsection*{Main responsible}
Emma Holmberg Ohlsson


\section{Task Descriptions}
\subsection*{Proposition}
Task descriptions are functional requirements.
\subsection*{Reason}
Task descriptions describe what the user and product do together.
\subsection*{Correct Answer}
D - Proposition is false, but the reason is a true statement.
\subsection*{Motivation}
The proposition is false since task descriptions only says what the product and user do together, and not what parts of a task that the product is responsible for. That means that it is a domain level requirement. Had we specified what the product should do it had been a product requirement.
\subsection*{Reference}
Lau: Chapter 3 pages 92, 99
\subsection*{Learning objective}
1.1.1, 1.1.2
\subsection*{Main responsible}
Emma Holmberg Ohlsson


\section{Focus Groups}
\subsection*{Proposition}
Focus groups are used mainly towards the end of a project in order to validate requirements together with the stakeholders.
\subsection*{Reason}
Focus groups are similar to brainstorming sessions where various stakeholders participate.
\subsection*{Correct Answer}
D - The proposition is wrong but the reason is a correct statement.
\subsection*{Motivation}
Focus groups are used early in the project as an elicitation technique, often to generate new ideas. They are not used for validating requirements late in a project. Typically end users and other stakeholders take part in the focus groups.
\subsection*{Reference}
Lau: Chapter 8, page 343, 352
\subsection*{Learning objective}
1.1.1, 1.1.3, 1.1.4
\subsection*{Main responsible}
Martin Richter


\section{Elicitation Barriers}
\subsection*{Proposition}
All elicitation barriers can be overcome.
\subsection*{Reason}
With the correct elicitation technique and the elicitation performed by a requirement specialist all the barriers can be conquered.
\subsection*{Correct Answer}
E - Both the proposition and the reason are false.
\subsection*{Motivation}
Elicitation barriers are existing problems. Some can be overcome by using different elicitation methods, but not all. For example for "lack of imagination" or "cannot express what they need" you can with elicitation methods make the barriers manageable but you cannot overcome them completely. Not all barriers are obvious and if you don't know your barriers you cannot overcome them.
\subsection*{Reference}
Lau: Chapter 8 pages 334-335
\subsection*{Learning objective}
1.1.1, 1.1.3
\subsection*{Main responsible}
Emma Holmberg Ohlsson

\end{document}