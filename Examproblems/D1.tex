\documentclass[a4paper]{article}
\usepackage[utf8]{inputenc}
\usepackage[english]{babel}
\begin{document}
\section{Problem 1 - Focus Groups}
\subsection{Proposition}
Focus groups are used mainly towards the end of a project in order to validate requirements together with the stakeholders.
\subsection{Reason}
Focus groups are similar to brainstorming sessions where various stakeholders participate.
\subsection{Correct Answer}
D, The proposition is wrong but the reason is a correct statement.
\subsection{Motivation}
Focus groups are used early in the project, often to generate new ideas. And not for validating requirements late in a project. Typically end users and other stakeholders take part in the focus groups.
\subsection{Reference}
Lau Chapter 8, page 343, 352
\subsection{Learning objective}
1.1.3
\subsection{Main responsible}
Martin Richter

\section{Problem 2 - Vivid Scenarios}
\subsection{Proposition}
Vivid scenarios should not be used as requirements
\subsection{Reason}
Vivid scenarios are too hard to verify to be good requirements, and they do not cover all task that the system should support.
\subsection{Correct Answer}
A
\subsection{Motivation}
Vivid scenarios does not cover all task that should be performed by the system, and it is hard to verify that they are fulfilled in the system during development. Therefore they are not suitable as requirements. On the other hand they can be useful to include in requirement documentation to give the developer a sense of how the system will be used.
\subsection{Reference}
Lau: Chapter 3 pages 114-115
\subsection{Learning objective}
1.1.1, 1.1.3
\subsection{Main responsible}
Emma Holmberg Ohlsson


\section{Problem 3 - Task Descriptions}
\subsection{Proposition}
Task descriptions are functional requirements
\subsection{Reason}
Task descriptions describe what the user and product do together
\subsection{Correct Answer}
D (Proposition is false, but the reason is a true statement.)
\subsection{Motivation}
The proposition is false as task descriptions only say what the product and user do together, and not what parts of a task that is the product’s responsibilities. That means that it is a domain level requirement. Had we specified what the product should do it had been a product requirement.
\subsection{Reference}
Lau: Chapter 3 pages 92, 99
\subsection{Learning objective}
1.1.1
\subsection{Main responsible}
Emma Holmberg Ohlsson


\section{Problem 4 - Context Diagram}
\subsection{Proposition}
A context diagram is only useful in a tender situation.
\subsection{Reason}
In a tender situation functional and design requirements are of none or little use . A context diagram describes the things that go on in the system
\subsection{Correct Answer}
D (Proposition is false, but the reason is a true statement.)
\subsection{Motivation}
Context diagrams can be useful in all sorts of projects to describe the domain.
\subsection{Reference}
Lau: Chapter 1 pages 20
\subsection{Learning objective}
??
\subsection{Main responsible}
Emma Holmberg Ohlsson


\section{Problem 5 - Data requirements}
\subsection{Proposition}
A data model and a data dictionary both need to exist if one of them does.
\subsection{Reason}
A data model describes what entities exist and their relations. A data dictionary describes the entities.
\subsection{Correct Answer}
D (Proposition is false, but the reason is a true statement.)
\subsection{Motivation}
Data model should have an accompanying data dictionary to explain it, but a data dictionary can exist on its own.
\subsection{Reference}
Lau: Chapter 2 pages 44-60
\subsection{Learning objective}
??
\subsection{Main responsible}
Emma Holmberg Ohlsson

\section{Problem 6 - Elicitation Barriers}
\subsection{Proposition}
All elicitation barrier can be overcome.
\subsection{Reason}
With the correct elicitation method and the elicitation performed by a requirement specialist all the barriers can be 
\subsection{Correct Answer}
E
\subsection{Motivation}
Elicitation barrier are existing problems. Some can be overcome by using different elicitation methods, but not all.
\subsection{Reference}
Lau: Chapter 8 pages 334-335
\subsection{Learning objective}

\subsection{Main responsible}
Emma Holmberg Ohlsson


\section{Problem 7 - Data expressions}
\subsection{Proposition}
Data expressions give a good overview for everyone involved in the project
\subsection{Reason}
It is easy to understand for  both developers and customers. 
\subsection{Correct Answer}

\subsection{Motivation}
Not a good overview for the customer and hard to administrate. Not easy to understand the whole.
\subsection{Reference}
Lau: Chapter 2 pages 60, 63-64
\subsection{Learning objective}
??
\subsection{Main responsible}
Alexander Nässlander

\section{Problem 8 - Asking suppliers}
\subsection{Proposition}
Asking suppliers can lead to fewer demands.
\subsection{Reason}

\subsection{Correct Answer}

\subsection{Motivation}
You can get to know what is standard or praxis so that you don't need to write demands for as much details.

\subsection{Reference}
Lau: Chapter 8 pages 346
\subsection{Learning objective}
??
\subsection{Main responsible}
Hannes Johansson

\end{document}