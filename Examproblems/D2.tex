\documentclass[a4paper]{article}
\usepackage[utf8]{inputenc}
\usepackage[english]{babel}
\title{Exam Questions 1}
\author{Group D\\ Emma Holmberg Ohlsson, Alexander Nässlander,\\Martin Richter, Hannes Johansson, Johan Mattsson,\\Philip Holgersson, Mattias Eklund, Fredrik Åkerberg}
\date{\today}

\begin{document}
	\maketitle
	\thispagestyle{empty}
	\setcounter{page}{0}
	\pagebreak

%\section{}
%\subsection*{Proposition}
%\subsection*{Reason}
%\subsection*{Correct Answer}
%\subsection*{Motivation}
%\subsection*{Reference}
%\subsection*{Learning objective}
%\subsection*{Main responsible}

\section{MDRE/RP/PRIO}
\subsection*{Proposition}
Considering the alfa/beta model of MDRE selection qualtiy. Alfa requirements are high quality requirements which should selected.
\subsection*{Reason}
When doing MDRE, it can be useful if the development organization can collaborate together with the marketing department.
\subsection*{Correct Answer}
B, both reason and proposition are correct but the reason does not explain the proposition.
\subsection*{Motivation}
The alfa requirements are the "golden grains" from the MDRE process as expressed in the text. And it is true that the marketing department can be an asset when doing MDRE as they typically know more about the market and the potential buyers. But the two statements are not correlated.
\subsection*{Reference}
Primarily MDRE sect 13.3.1 and 13.2.2.
\subsection*{Learning objective}
1.1.1 \& 1.1.6
\subsection*{Main responsible}
Martin Richter


\section{}
\subsection*{Proposition}
\subsection*{Reason}
\subsection*{Correct Answer}
\subsection*{Motivation}
\subsection*{Reference}
Lau: Chapter 6 page
\subsection*{Learning objective}
\subsection*{Main responsible}

\end{document}