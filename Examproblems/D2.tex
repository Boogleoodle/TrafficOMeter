\documentclass[a4paper]{article}
\usepackage[utf8]{inputenc}
\usepackage[english]{babel}
\title{Exam Questions 2}
\author{Group D\\ Emma Holmberg Ohlsson, Alexander Nässlander,\\Martin Richter, Hannes Johansson, Johan Mattsson,\\Philip Holgersson, Mattias Eklund, Fredrik Åkerberg}
\date{\today}

\begin{document}
	\maketitle
	\thispagestyle{empty}
	\setcounter{page}{0}
	\pagebreak

\section{Writing a proposal}
\subsection*{Proposition}
Consider that you are writing a proposal to your client, then it is beneficial if the client has given you little information as it will force you to visit the client and get more domain knowledge.  
\subsection*{Reason}
A visit to the costumer is a good idea if she/he has not given you any background information about why she/he needs the system. Then you can get a clearer view of why she/he needs that system
\subsection*{Correct Answer}
D, the proposition is wrong but the reason is a correct statement.
\subsection*{Motivation}
The reason is correct as visiting the client often increases the domain knowledge. Even if it is good to visit the client, it is NOT beneficial if they have given you so little information that it forces you to do so.
\subsection*{Reference}
Lau:7, p.308-312
\subsection*{Learning objective}
1.1.3 
\subsection*{Main responsible}
Alexander Nässlander

\section{Quality Requirements}
\subsection*{Proposition}
Correctness is a very important quality factor for systems operating in high risk situations, such as in outer space.
\subsection*{Reason}
Correctness is the measure of how accurately the system performs tasks and delivers information.
\subsection*{Correct Answer}
C, the proposition is true, but the reason is false.
\subsection*{Motivation}
When operating in high risk environments it is vital that there are no errors in the system, considering it’s a situation of life and death. Correctness is however not a measurement of how accurately a system performs tasks, but rather a measurement of how many errors it contains.
\subsection*{Reference}
Lau: 6, p. 220
\subsection*{Learning objective}
1.1.1, 1.1.3
\subsection*{Main responsible}
Johan Mattsson

\section{Quality Requirements}
\subsection*{Proposition}
The different breakpoints (utility, differentiation, saturation) in the Quper-model are based on the current market expectations.
\subsection*{Reason}
The competing products are placed in the model as a reference
\subsection*{Correct Answer}
B: Both the proposition and the reason are correct statements, BUT the reason does not explain the proposition.
\subsection*{Motivation}
The breakpoints are based on market expectation and not directly linked to competing products, they are linked to what is acceptable in the current marked. The competing products are also used in the Quper-model but they have their own dots.
\subsection*{Reference}
Quper article: p.45
\subsection*{Learning objective}
1.1.3, 1.1.6
\subsection*{Main responsible}
Hannes Johansson

\section{Validation}
\subsection*{Proposition}
It is not necessary for developers and clients to review the tasks.
\subsection*{Reason}
Tasks are only there for providing context to the requirements.
\subsection*{Correct Answer}
E, both the proposition and the reason are false.
\subsection*{Motivation}
The people writing the specification might have missed requirements that emerges when reviewing the tasks. Tasks are not only for providing context but are also requirements themselves (task requirements).
\subsection*{Reference}
Lau: 9, p. 382-385
\subsection*{Learning objective}
1.1.1, 1.1.4
\subsection*{Main responsible}
Mattias Eklund

\section{Checking and Validation}
\subsection*{Proposition}
The CRUD-matrix is used to find missing quality requirements
\subsection*{Reason}
The CRUD-matrix helps to check the requirement specification for inconsistencies 
\subsection*{Correct Answer}
D, the proposition is false, but the reason is a true statement.
\subsection*{Motivation}
The CRUD-matrix checks for consistencies in the functional context. Which means that it does nothing for quality requirements
\subsection*{Reference}
Lau: 9, p. 386
\subsection*{Learning objective}
1.1.1, 1.1.2, 1.1.4
\subsection*{Main responsible}
Emma Holmberg Ohlsson

\section{Market Driven Requirements Engineering}
\subsection*{Proposition}
Considering the alfa/beta model of MDRE selection quality. Alfa requirements are high quality requirements which should be selected.
\subsection*{Reason}
When using MDRE, it can be useful if the development organization can collaborate together with the marketing department.
\subsection*{Correct Answer}
B, both reason and proposition are correct but the reason does not explain the proposition.
\subsection*{Motivation}
The alfa requirements are the "golden grains" from the MDRE process as expressed in the text. And it is true that the marketing department can be an asset when doing MDRE as they typically know more about the market and the potential buyers. But the two statements are not correlated.
\subsection*{Reference}
MDRE article: sect 13.3.1 and 13.2.2.
\subsection*{Learning objective}
1.1.1, 1.1.6
\subsection*{Main responsible}
Philip Holgersson

\section{Market Driven Requirements Engineering}
\subsection*{Proposition}
A company using MDRE should define its processes before expanding its operations.
\subsection*{Reason}
Without well defined processes it is hard to transfer knowledge between personnel. 
\subsection*{Correct Answer}
A, both the proposition and the reason are true. Also, the reason explains the proposition.
\subsection*{Motivation}
Companies without a defined process take a significant risk if key persons leave the organization, since they lack the necessary documentation and structure. In times of downsizing or rapid expansion it is very difficult to install a repeatable process.
\subsection*{Reference}
MDRE: sect 13.2.3
\subsection*{Learning objective}
1.1.5, 1.1.6
\subsection*{Main responsible}
Fredrik Åkerberg
 
\section{Agile Requirements Engineering}
\subsection*{Proposition}
Some quality requirements, such as security and scalability, have a tendency to be neglected when using agile RE. 
\subsection*{Reason}
When working with agile RE the prioritization is often based on the product owner's business value, which is rarely defined as high security or scalability.
\subsection*{Correct Answer}
A, the proposition is correct and the reason explains why.
\subsection*{Motivation}
The article on AGRE brings up the neglect of quality requires as a challenge with AGRE. It also says that the increased interaction with the product owner leads to prioritization decisions to be based on business value.
\subsection*{Reference}
AGRE article: p. 64
\subsection*{Learning objective}
1.1.3, 1.1.6
\subsection*{Main responsible}
Martin Richter

\end{document}
