\documentclass[a4paper]{article}
\usepackage[utf8]{inputenc}
\usepackage[english]{babel}
\title{Exam Questions 2}
\author{Group D\\ Emma Holmberg Ohlsson, Alexander Nässlander,\\Martin Richter, Hannes Johansson, Johan Mattsson,\\Philip Holgersson, Mattias Eklund, Fredrik Åkerberg}
\date{\today}

\begin{document}
	\maketitle
	\thispagestyle{empty}
	\setcounter{page}{0}
	\pagebreak

%\section{}
%\subsection*{Proposition}
%\subsection*{Reason}
%\subsection*{Correct Answer}
%\subsection*{Motivation}
%\subsection*{Reference}
%\subsection*{Learning objective}
%\subsection*{Main responsible}

\section{MDRE/RP/PRIO}
\subsection*{Proposition}
Considering the alfa/beta model of MDRE selection qualtiy. Alfa requirements are high quality requirements which should selected.
\subsection*{Reason}
When doing MDRE, it can be useful if the development organization can collaborate together with the marketing department.
\subsection*{Correct Answer}
B, both reason and proposition are correct but the reason does not explain the proposition.
\subsection*{Motivation}
The alfa requirements are the "golden grains" from the MDRE process as expressed in the text. And it is true that the marketing department can be an asset when doing MDRE as they typically know more about the market and the potential buyers. But the two statements are not correlated.
\subsection*{Reference}
Primarily MDRE article sect 13.3.1 and 13.2.2.
\subsection*{Learning objective}
1.1.1 \& 1.1.6
\subsection*{Main responsible}
Martin Richter

\section{Release Planning}
\subsection*{Proposition}
Release planning is best done with the science of release planning.
\subsection*{Reason}
The art of release planning is biased since it relies only on human judgement and therefore should not be a part of the release planning.
\subsection*{Correct Answer}
E - Both the proposition and the reason are false.
\subsection*{Motivation}
After reading the article I consider the best way to be release planning using the hybrid approach, where you start with the science of release planning to get a couple of possible good release plans, but then use the art of release planning to choose and make minor alterations to get the most appropriate one. People in the project are good for the judgement calls that an algorithm cannot handle and therefore the art of science should also be included in the release planning. 
\subsection*{Reference}
RP article - The art and science of software release planning
\subsection*{Learning objective}
1.1.1, 1.1.3, 1.1.4
\subsection*{Main responsible}
Emma Holmberg Ohlsson

\section{QUPER}
\subsection*{Proposition}
QUPER is used to see to what degree it is worth to spend time on a quality factor.
\subsection*{Reason}
QUPER gives an overview of the cost and benefit view so that it becomes clear when it is not worth to improve a quality factor since the cost is to high or the benefit to low.
\subsection*{Correct Answer}
A - Both the proposition and the reason are correct statements,
AND the reason explains the proposition in a correct way. 
\subsection*{Motivation}
It is hard to get an overview of how much better the product gets if we make a part just a little faster, if it is worth it form a cost/benefit perspective. With QUPER we can see where the cost-barriers appear and when we pass breakpoints that tell us if we do enough or too much. With the help of these breakpoints and barriers we can determine how much time should be spent on the quality feature and when we have done enough.
\subsection*{Reference}
QUPER article - Supporting Roadmapping of Quality Requirements
\subsection*{Learning objective}
1.1.1, 1.1.3
\subsection*{Main responsible}
Emma Holmberg Ohlsson




\section{INTDEP}
\subsection*{Proposition}
Requirements interdependencies is important when in comes to release planning and they are often identified explicitly.
\subsection*{Reason}
The amount of requirement interdependencies makes them easy to identify and the idientification become less complex as the numbers increase.
\subsection*{Correct Answer}
Båda är fel
\subsection*{Motivation}
Requirements interdependencies has an important role in release planning, but they are NOT often identified explicitly because it becomes harder to identify them and the complexity grows as the numbers increase.
\subsection*{Reference}
INTDEP 3.3
\subsection*{Learning objective}
1.1.1 \&\& 1.1.3? eller 1.1.4
\subsection*{Main responsible}
Philip Holgersson



\end{document}