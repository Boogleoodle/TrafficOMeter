\documentclass[a4paper]{article}
\usepackage[utf8]{inputenc}
\usepackage[english]{babel}
\title{Exam Questions 2}
\author{Group D\\ Emma Holmberg Ohlsson, Alexander Nässlander,\\Martin Richter, Hannes Johansson, Johan Mattsson,\\Philip Holgersson, Mattias Eklund, Fredrik Åkerberg}
\date{\today}

\begin{document}
	\maketitle
	\thispagestyle{empty}
	\setcounter{page}{0}
	\pagebreak

%\section{}
%\subsection*{Proposition}
%\subsection*{Reason}
%\subsection*{Correct Answer}
%\subsection*{Motivation}
%\subsection*{Reference}
%\subsection*{Learning objective}
%\subsection*{Main responsible}


%Ch 5,7 : 1 question
\section{Writing a proposal}
\subsection*{Proposition}
If you are about to write a proposal to your costumer, and the costumer has given you insufficient information in order for you to give a full answer. Instead of making assumptions of the system, a visit to that costumer will eliminate assumptions.   
\subsection*{Reason}
A visit to the costumer is a good idea if she/he has not given you any background information about why she/he needs the system.
Then you can get a clearer view of why she/he needs that system
\subsection*{Correct Answer}

\subsection*{Motivation}
 ??
\subsection*{Reference}

\subsection*{Learning objective}

\subsection*{Main responsible}
Alexander Nässlander


% Ch6 & QUPER, 2 problems
\section{QUPER}
\subsection*{Proposition}
QUPER is used to see to what degree it is worth to spend time on a quality factor.
\subsection*{Reason}
QUPER gives an overview of the cost and benefit view so that it becomes clear when it is not worth to improve a quality factor since the cost is to high or the benefit to low.
\subsection*{Correct Answer}
A - Both the proposition and the reason are correct statements,
AND the reason explains the proposition in a correct way. 
\subsection*{Motivation}
It is hard to get an overview of how much better the product gets if we make a part just a little faster, if it is worth it form a cost/benefit perspective. With QUPER we can see where the cost-barriers appear and when we pass breakpoints that tell us if we do enough or too much. With the help of these breakpoints and barriers we can determine how much time should be spent on the quality feature and when we have done enough.
\subsection*{Reference}
QUPER article - Supporting Roadmapping of Quality Requirements
\subsection*{Learning objective}
1.1.1, 1.1.3
\subsection*{Main responsible}
Emma Holmberg Ohlsson


\section{LAU6}
\subsection*{Proposition}
Correctness is a very important quality factor for systems operating in high risk situations, such as in outer space.
\subsection*{Reason}
Correctness is the measure of how accurately the system performs tasks and delivers information.
\subsection*{Correct Answer}
C - The proposition is true, but the reason is false.
\subsection*{Motivation}
When operating in high risk environments it is vital that there are no errors in the system, considering it’s a situation of life and death. Correctness is however not a measurement of how accurately a system performs tasks, but rather a measurement of how many errors it contains.
\subsection*{Reference}
LAU chap 6
\subsection*{Learning objective}
page 220
\subsection*{Main responsible}
Johan Mattsson

\section{Quality Requirements}
\subsection*{Proposition}
The cost to benefit ratio is hard to figure out when trying to determinate boundaries e.g the response time for a feature
%for a calculation done by a system when the requirement does not have a physical explination
\subsection*{Reason}
For different systems and features there are different factors that one has to take into account which make it hard to decided about the trade off between cost and benefit.
\subsection*{Correct Answer}
A - Both the Proposition and the Reason is true
\subsection*{Motivation}
If the requirement is not critical in the since that the system would be unusable if a specific response time is held, then there is a trade off between the cost and the benefit for setting a certain demand for the response time. When many different factors are in play this could be ha hard decision to take. \textbf{här borde stå något mer}
\subsection*{Reference}
Lau chapter 6, section 6.3 pages 228 - 233
\subsection*{Learning objective}
1.1.1 \& 1.1.3
\subsection*{Main responsible}
Philip Holgersson

\section{Quality Requirements}
\subsection*{Proposition}
The cost to benefit ratio can be hard to determine for a function that is used in several different systems.
\subsection*{Reason}
The function might be used with different frequency for different systems and does not have the same benefit in all the systems.
\subsection*{Correct Answer}
A - Both the Proposition and the Reason is true
\subsection*{Motivation}
If a function is used once a week in one system and once a day in another, the later system might have higher demands on e.g the response time. The first system might be okay with a response time of 5 minutes and with a lower cost while the latter wants it to go much faster. 
\subsection*{Reference}
Lau chapter 6, section 6.3 pages 228 - 233
\subsection*{Learning objective}
1.1.1 \& 1.1.3
\subsection*{Main responsible}
Philip Holgersson

\section{Quality Requirements}
\subsection*{Proposition}
The different breakpoints (Utility, differentiation, saturation) in the Quper-model are based on the current market expectations.
\subsection*{Reason}
The competing products are placed in the model as a reference
\subsection*{Correct Answer}
B: Both the proposition and the reason are correct statements, BUT the reason does not explain the proposition.
\subsection*{Motivation}
The breakpoints are based on market expectation and not directly linked to competing products, they are linked to what is acceptable in the current marked. The competing products are also used in the Quper-model but they have their own dots.
\subsection*{Reference}
Quper: "45"
\subsection*{Learning objective}
??
\subsection*{Main responsible}
Hannes Johansson


%Ch 9 & INSP: 2 problems
\section{Checking and Validation}

\subsection*{Proposition}
The main idea of checking and validating the requirements is to make sure that the requirements are understandable by the costumer and that it meets the costumers needs.
\subsection*{Reason}
By checking the qualities of the requirements and making sure that the requirements on any level can be traced to the domain requirements and goal mission, the costumer needs can be met in a good manner.
\subsection*{Correct Answer}

\subsection*{Motivation}
 It is good to make it understandable for the costumer and meet its needs, but the requirements must also be understandable for the developers as well. If the developers has any hesitations about the requirements, it is going to be hard to develop the right product for the costumer.
\subsection*{Reference}
Lau chapter 9
\subsection*{Learning objective}

\subsection*{Main responsible}
Alexander Nässlander



\section{Checking and Validation}
\subsection*{Proposition}
The CRUD-matrix is used to find missing quality requirements

\subsection*{Reason}
The CRUD-matrix helps to check the requirement specification for inconsistencies 
\subsection*{Correct Answer}
D: The proposition is false, but the reason is a true statement.
\subsection*{Motivation}
The CRUD-matrix checks for consistencies in the functional context. Which means that it does nothing for quality requirements
\subsection*{Reference}
Lau: 386
\subsection*{Learning objective}
??
\subsection*{Main responsible}
Hannes Johansson

\section{LAU9}
\subsection*{Proposition}
It’s not necessary for developers and customers to review the part of the spec concerning tasks.
\subsection*{Reason}
The tasks part of the spec is just the other requirements in the spec put into context.
\subsection*{Correct Answer}
E - Both the proposition and the reason are false.
\subsection*{Motivation}
The people writing the spec might have missed requirements that emerges when reviewing the tasks.
\subsection*{Reference}
LAU chap 9
\subsection*{Learning objective}
s382-385
\subsection*{Main responsible}
Johan Mattsson

\section{INSP}
\subsection*{Proposition}
During the correction process during inspection, all defects are corrected.
\subsection*{Reason}
 It is easy to know exactly what needs to be corrected, since everything is well-stuctured and already identified.
\subsection*{Correct Answer}
E - both the proposition and the reason are false
\subsection*{Motivation}
Eventhough everything is well-structed and identified, it's likely that something slips through and therefore it's not garanteed that every defekt will be corrected.
There might also exist false positives, which must be dealt with.
\subsection*{Reference}
INSP: 3.6.1.6
\subsection*{Learning objective}
???????
\subsection*{Main responsible}
Fredrik Åkerberg



%MDRE, PRIO & RP: 2 problems
\section{Market Driven Requirements Engineering}
\subsection*{Proposition}
Considering the alfa/beta model of MDRE selection qualtiy. Alfa requirements are high quality requirements which should selected.
\subsection*{Reason}
When doing MDRE, it can be useful if the development organization can collaborate together with the marketing department.
\subsection*{Correct Answer}
B, both reason and proposition are correct but the reason does not explain the proposition.
\subsection*{Motivation}
The alfa requirements are the "golden grains" from the MDRE process as expressed in the text. And it is true that the marketing department can be an asset when doing MDRE as they typically know more about the market and the potential buyers. But the two statements are not correlated.
\subsection*{Reference}
Primarily MDRE article sect 13.3.1 and 13.2.2.
\subsection*{Learning objective}
1.1.1 \& 1.1.6
\subsection*{Main responsible}
Martin Richter



\section{Release Planning}
\subsection*{Proposition}
Release planning is best done with the science of release planning.
\subsection*{Reason}
The art of release planning is biased since it relies only on human judgement and therefore should not be a part of the release planning.
\subsection*{Correct Answer}
E - Both the proposition and the reason are false.
\subsection*{Motivation}
After reading the article I consider the best way to be release planning using the hybrid approach, where you start with the science of release planning to get a couple of possible good release plans, but then use the art of release planning to choose and make minor alterations to get the most appropriate one. People in the project are good for the judgement calls that an algorithm cannot handle and therefore the art of science should also be included in the release planning. 
\subsection*{Reference}
RP article - The art and science of software release planning
\subsection*{Learning objective}
1.1.1, 1.1.3, 1.1.4
\subsection*{Main responsible}
Emma Holmberg Ohlsson

\section{MDRE}
\subsection*{Proposition}
MDRE doesn't apply well to companies with an undefined process.
\subsection*{Reason}
If key persons leave the company, key knowledge might also leave.
\subsection*{Correct Answer}
A - both the proposition and the reason are true
\subsection*{Motivation}
Companies without a defined process take a significant risk if key persons leave the organization, since they lack the necessary documentation and structure. In times of downsizing or rapid expansion it is very difficult to install a repeatable proces
\subsection*{Reference}
MDRE: 13.2.3
\subsection*{Learning objective}
?????
\subsection*{Main responsible}
 Fredrik Åkerberg
 
 

% AGRE & INTDEP, 1 problem
\section{Agile Requirements Engineering}
\subsection*{Proposition}
Non functional requirements, such as security and scalability, have a tendency to be neglected when using agile RE. 
\subsection*{Reason}
When working with agile RE the prioritization is often based on the product owner's business value, which often is defined as high usability or more features.
\subsection*{Correct Answer}
A, the proposition is correct and the reason explains why.
\subsection*{Motivation}
The article on AGRE brings up the neglect of non functional requires as a challenge with AGRE. It also says that the increased interaction with the product owner leads to prioritization decisions to be based on business value.
\subsection*{Reference}
AGRE article p. 64
\subsection*{Learning objective}
1.1.3 \& 1.1.6
\subsection*{Main responsible}
Martin Richter

\section{INTDEP}
\subsection*{Proposition}
Requirements interdependencies is important when in comes to release planning and they are often identified explicitly.
\subsection*{Reason}
The amount of requirement interdependencies makes them easy to identify and the idientification become less complex as the numbers increase.
\subsection*{Correct Answer}
E - Both the Proposition and the Reason are false
\subsection*{Motivation}
Requirements interdependencies has an important role in release planning, but they are NOT often identified explicitly because it becomes harder to identify them and the complexity grows as the numbers increase.
\subsection*{Reference}
INTDEP 3.3
\subsection*{Learning objective}
1.1.1 \&\& 1.1.3? eller 1.1.4
\subsection*{Main responsible}
Philip Holgersson


\end{document}
